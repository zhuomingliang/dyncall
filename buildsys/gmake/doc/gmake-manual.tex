\documentclass{article}

\newcommand{\var}[1]{{\tt #1}}
\newcommand{\val}[1]{{\tt #1}}
\newcommand{\file}[1]{{\tt #1}}

\begin{document}

\title{buildsys/gmake}
\author{Daniel Adler}
\maketitle

\section{Usage}

\subsection{Building}

Before building a project, one has to configure the project.

\begin{verbatim}
./configure
.\configure.bat
gmake or make
\end{verbatim}

The configure script creates a top-level \file{ConfigRules} that
specifies all \var{BUILD\_*} and \var{INSTALL\_*} variables.

Here's a sample \file{ConfigRules} file:

\begin{verbatim}
BUILD_OS=windows
BUILD_ARCH=x86
BUILD_TOOL=gcc
BUILD_ASM=as
INSTALL_DIR=/usr/local
\end{verbatim}

\subsection{Project setup}

Write up a \file{GNUmakefile} file in some project folder. Specify the
relative path to the top-level project directory which contains the \file{buildsys} folder
using the \var{TOP} variable. Your project settings are embedded between
two includes, namely the prolog and the epilog files.

\begin{verbatim}
# Sample GNUmakefile for buildsys/gmake
TOP	= ../..
include $(TOP)/buildsys/gmake/prolog.gmake
# ... project specific variables ...
include $(TOP)/buildsys/gmake/epilog.gmake
\end{verbatim}

\section{Configuration variables}
  
\subsection{BUILD\_OS}

Specifies the target operating system or embedded run-time platform:

\begin{tabular}{ll}
Value         & Description           \\
\hline
\val{windows} & Microsoft Windows     \\
\val{darwin}  & Mac OS                \\
\val{linux}   & Linux                 \\
\val{sunos}   & SunOS                 \\
\val{psp}     & Playstation Portable  \\
\val{nds}     & Nintendo DS           \\
\end{tabular}

\subsection{BUILD\_ARCH}

Specifies the processor architecture:

\begin{tabular}{ll}
Value        & Description                 \\
\hline
\val{x86}    & X86 32-bit Architecture     \\
\val{x64}    & X86 64-bit Architecture     \\
\val{ppc32}  & PowerPC 32-bit Architecture \\
\val{arm9e}  & ARM9E Architecture          \\
\val{mips32} & MIPS 32-bit Architecture    \\
\end{tabular}

\subsection{BUILD\_TOOL}

Specifies the build tool chain:

\begin{tabular}{ll}
Value      & Description             \\
\hline
\val{gcc}  & GNU Compiler Collection \\
\val{msvc} & Microsoft Visual C++ (windows) \\
\end{tabular}

\subsection{BUILD\_ASM}

Specifies the assembler tool:

\begin{tabular}{ll}
Value      & Description              \\
\hline
\val{as}   & GNU Assembler            \\
\val{nasm} & NASM Assembler (x86/x64) \\
\val{ml}   & Micrsoft Macro Assembler (x86/x64)\\
\end{tabular}

\subsection{Others}

\begin{tabular}{ll}
Variable name		& Description		\\
\hline
\var{INSTALL\_DIR}	& Installation directory \\
\end{tabular}
\section{OS-specific variables}

\begin{tabular}{ll}
Variable Name    & Description                   \\
\hline
\var{APP\_SUFFIX} & application suffix            \\
\var{LIB\_SUFFIX} & static library suffix         \\
\var{DLL\_SUFFIX} & dynamic linked library suffix \\
\end{tabular}

\section{Tool-specific variables}

\begin{tabular}{ll}
Variable Name    & Description                   \\
\hline
\var{OBJ\_SUFFIX} & object file suffix            \\
\end{tabular}


\section{Misc}

\begin{tabular}{ll}
Variable name		& Description		\\
\hline
\var{PDF}		& build pdf manual (sources in latex) \\
\end{tabular}

\section{Project variables}

\begin{tabular}{ll}
Variable name            & Description                       \\
\hline
\var{TOP}		 & relative path to top directory    \\
\var{DIRS}		 & sub-project directory names	     \\
\var{MODS}               & list of module names to link      \\
\var{APP}                & build application name            \\
\var{LIBRARY}            & build static library name         \\
\var{DLL}                & build dynamic library name        \\
\var{LINKER}             & specify application linker, values: \val{c} or \val{cxx} otherwise default linker\\
\var{LINK\_PATHS}	 & list of library paths \\
\var{LINK\_LIBS}		 & list of library names \\
\var{USE\_CXX\_EXCEPTIONS} & 0 or 1 to use C++ Exceptions                \\
\var{USE\_CXX\_RTTI}       & 0 or 1 to use C++ Run-time Type Information \\
\var{CPPFLAGS}		   & C Preprocessor flags (e.g. -Iincludename) \\
\var{CFLAGS}		& C Compiler flags \\
\var{CXXFLAGS}		& C++ Compiler flags \\
\var{INSTALL\_APP}	& 0 or 1 to install executable to bin sub-directory \\
\var{INSTALL\_LIB}	& 0 or 1 to install library to lib sub-directory \\
\var{INSTALL\_INCLUDES} & List of header files to copy to include sub-directory \\
\end{tabular}

\section{OS notes}

\begin{description}
\item [windows] gcc and msvc tool-chain support
\item [psp] gcc only
\end{description}

\end{document}

